\section{\textbf{Setting up the development environment }}\label{sec:relatedwork}

\subsection{Dynamic ad-insertion and content orchestration workflows through manifest manipulation in HLS and MPEG-DASH}

The work presented in \cite{adinsertion} uses Manifest manipulation to dynamically insert ads into media content with the FAMIUM DAI solution. This study mentions how HLS specifies a so-called \#EXT-X-DISCONTINUITY tag, which is used in the manifest file to decorate a generic discontinuity at the source. This could be a switch of e.g. encoding parameters, input sources, or an advertisement, which has been spliced into the source. This was useful in our work to concatenate segments from different video sources, which will be explained in more detail in the Our approach section

This work has a component named The Manifest Stitcher, which performs individual manifest stitching, based on content playlists served by another entity in their architecture. The component creates MPEG-DASH and HLS manifests on the fly. The Stitcher requests the manifest files for the original content assets as well as for the pre-conditioned ad asset and creates a new manifest comprising multiple periods. This new manifest is sent to the client and allows him to play the individual stitched content.

The basic idea behind that approach is to ingest content or ads into DASH and HLS Streams by repackaging the stream. The output stream behaves like a linear stream, just like our use case for a single stream containing multiple input video streams. The re-packaged content can be handled by the player in the same way as the linear original stream. To guarantee smooth playback without re-initialization of the video player, it is necessary to condition the original stream and the content to be inserted with the same content specifications in terms of audio and video coding settings. These last conditions are also considered for our work, streams with different audio formats will not play properly.
 
\subsection{ABR streaming with separate audio and video tracks: measurements and best practices}

This paper \cite{measures} examines the state of the art in the handling of demuxed audio and video tracks in predominant Adaptive Bitrate Streaming protocols (DASH and HLS). They found several limitations in existing practices both in the protocols and the player implementations, which can cause undesirable behaviors such as stalls, selection of potentially undesirable combinations such as very low-quality video with very high-quality audio, etc. Our work had to handle streams with separate audio tracks, having to join both video and audio segments in separate manifest files without consideration on how well is the quality of experience; this matter will be covered in the Evaluation and Results section.

\subsection{Increasing ad personalization with server-side ad insertion}

This paper \cite{ssai} examines the architectures required to achieve server-side advertisement insertion so that multiple concurrent advertising manifests can be delivered in a timely fashion. Cloud and cloud-assisted software solutions were required. 

When a video plays, the player makes the necessary calls to obtain the next chunk of content to be shown from the manifest. In a well-designed player, where the segments come from the same source, there should be a seamless display and a reasonable quality of experience. What is required, then, is a method of inserting targeted advertising into individual delivery paths that provides clear metrics, protects against advertising blocking or skipping, and maintains a consistent quality of experience for the consumer. The solution lies in upstream insertion of the advertising so that a continuous stream arrives at the consumer device eliminating any possibility of discrimination between content and commercials, and avoiding freezes, black screens, and spinning wheels.

In \cite{ssai}, is shown an HLS example manifest that uses the "\#EXT-X-CUE-OUT" manifest declaration, which is the signal to the player that this is a commercial break. When using client-side advertisement insertion, the player will have some sort of logic to initiate a call to get the advertisements. The player will probably download ads to play within the time between the CUE markers. During the "\#EXT-X-CUE" breaks, it is clear that some calls are not to the content provider but advertising servers. By moving the advertising insertion to the server side and within the packaging process, the content provider address and the advertisement server provider address differences are eliminated and made to look the same from a manifest perspective. There is no need to put the \#EXT-X-CUE tags anymore as the ad insertion or replacement is already done. The client's content and the ads would be hosted at the same address. 

This eliminates the prospect of skipping the advertising, a consistent stream of content is also presented in the same resolution, codec, and encryption, which maintains the quality of experience. Finally, a single contiguous stream is sent to the client.